\documentclass[10pt]{article}

% Lines beginning with the percent sign are comments
% This file has been commented to help you understand more about LaTeX

% DO NOT EDIT THE LINES BETWEEN THE TWO LONG HORIZONTAL LINES

%---------------------------------------------------------------------------------------------------------

% Packages add extra functionality.
\usepackage{times,graphicx,epstopdf,fancyhdr,amsfonts,amsthm,amsmath,algorithm,algorithmic,xspace,hyperref}
\usepackage[left=1in,top=1in,right=1in,bottom=1in]{geometry}
\usepackage{sect sty}	%For centering section headings
\usepackage{enumerate}	%Allows more labeling options for enumerate environments 
\usepackage{epsfig}
\usepackage[space]{grffile}
\usepackage{booktabs}
\usepackage{forest}
\usepackage{enumitem}   
\usepackage{fancyvrb}
\usepackage{todonotes}

% This will set LaTeX to look for figures in the same directory as the .tex file
\graphicspath{.} % The dot means current directory.

\pagestyle{fancy}

\lhead{Final Project}
\rhead{\today}
\lfoot{CSCI 334: Principles of Programming Languages}
\cfoot{\thepage}
\rfoot{Spring 2024}

% Some commands for changing header and footer format
\renewcommand{\headrulewidth}{0.4pt}
\renewcommand{\headwidth}{\textwidth}
\renewcommand{\footrulewidth}{0.4pt}

% These let you use common environments
\newtheorem{claim}{Claim}
\newtheorem{definition}{Definition}
\newtheorem{theorem}{Theorem}
\newtheorem{lemma}{Lemma}
\newtheorem{observation}{Observation}
\newtheorem{question}{Question}

\setlength{\parindent}{0cm}

%---------------------------------------------------------------------------------------------------------

% DON'T CHANGE ANYTHING ABOVE HERE

% Edit below as instructed

\usepackage{array}

\title{CAT Language Specification} % Replace SnappyLanguageName with your project's name

\author{Michael Faulkner}

\begin{document}
  
\maketitle

\subsection*{Introduction}

CAT (Calculus and Algebra Tutor) is a domain-specific language designed to assist 
individuals with mathematical calculations. More specifically, the language allows 
users to input mathematical expressions and operations to perform on the expressions. 
Running a program in this language will then produce a textual represenation of the 
ensuing calculation as well as an optional \LaTeX{} file and PDF that displays the calculation
in a pretty manner.

\bigskip
While there exist computational tools for generalized mathematical expressions 
(for example Mathematica), such tools tend to focus only on getting to an answer. 
The purpose of CAT is to assist its users in understanding \textit{how} to get 
the answer by outputting the intermediate steps of the calculation. This 
addresses another point of annoyance with tools like Mathematica that will 
sometimes output, seemingly arbitrary, restrictions on the parameters alongside 
its final expression. By working through the intermediate steps of the calculation, 
CAT should allow its users to understand why such restrictions were imposed and 
what the limits of the calculation would be if such assumptions were not made.

\subsection*{Design Principles}

The primary goal of CAT is to produce easily understandable calculations. 
This goal corresponds with a prioritization of simpler techniques that are
easier to follow over more powerful techniques that might be able to find
more general answers but are less decipherable. This likely also means a
tradeoff in terms of speed as it is necessary to work through the intermediate
details of a calculation sufficiently to make them human readable.

\subsection*{Examples}

\begin{enumerate}
    \item The following example illustrates how each line of a CAT program is
    treated separately so long as there are no assignments. The first line would
    simplify to $1$, and the second line would simplify to $2x$. \\
    \texttt{(x+1)(x-1) - x\^{}2 \\ Differentiate x\^{}2 + y\^{}2 wrt x}

    \item The following example illustrates a function definition and how it
    might be used on later lines. Note that the particular variables used in
    the function definition have no affect on what variables it is passed later.
    This example should expand the term inside the integral to $x + y + x + x 
    = 3x + y$. Then, after performing the integration should find 
    $\left. \frac{3}{2}x^2+xy \right|_0^1 = \frac{3}{2} + y$ \\
    \texttt{z(x,y) = x + y \\ Integrate from 0 to 1 z(x,y) + z(x,x) wrt x}

    \item This last example highlights some synonyms for particular language
    constructs and additionally showcases how undefined variables will be treated
    as any symbolic variable in a mathematical expression would be. Evaluating the
    first integral requires splitting the integral across the discontinuity in the 
    function. And evaluating the derivative in the second line requires symbolically
    representing the derivative of y with respect to x. \\
    \texttt{Int (x\^{}2-x-6)/(x-3) from a to b wrt x \\ 
    Differentiate (x\^{}2+2xy-y\^{}2)/(z-3) with respect to x, where y(x)}
\end{enumerate}

\subsection*{Language Concepts}

At it's core, CAT is a way of expressing and manipulating mathematical expressions. 
Each line of a CAT program constitutes an expression that will be evaluated. Evaluating
an expression involves reducing it to the simplist form possible by applying operators
and performing algebraic manipulations. The primitives of CAT are variables (i.e. $x, y, z$) 
and numbers (i.e. $1, -10, 3.141592$). These primitives can be combined with mathematical
operations such as +, -, *, /, \^{}, $\sin$, $\cos$, $\log$, sqrt, Differentiate,
and Integrate in order to form expressions. Additionally, users are able to define variables
and functions (i.e. $a = \text{sqrt}(2)$, or $f(x)=x+1$) that can be used in later lines.

\subsection*{Formal Syntax}

\begin{tabular}{>{\texttt\bgroup}l<{\egroup} >{\texttt\bgroup}r<{\egroup} >{\texttt\bgroup}l<{\egroup}}
    <instruction> &::=& <expr> \\
    &|& <expr> `\textbackslash{}n' <instruction> \\
    <expr> &::=& <ws>\textsuperscript{*}<parens><ws>\textsuperscript{*} \\
    &|& <ws>\textsuperscript{*}<literal><ws>\textsuperscript{*} \\
    <parens> &::=& (<expr>) \\
    <literal> &::=& <number> \\
    &|& <variable> \\
    <number> &::=& <digits> \\
    &|& -<digits> \\
    &|& <digits>.<digits> \\
    &|& -<digits>.<digits> \\
    <digits> &::=& <d><digits> \\
    &|& <d> \\
    <d> &::=& 0 | 1 | 2 | 3 | 4 | 5 | 6 | 7 | 8 | 9 \\
    <variable> &::=& a | b | c |\dots| x | y | z \\
    <ws> &::=& Any non-newline whitespace character.
\end{tabular}

\subsection*{Semantics}

\begin{enumerate}[label= (\roman*)]
    \item The primitives in CAT are numbers and variables (i.e. $1.234$ or $x$)
    \item CAT supports many combining forms. These combining forms correspond with
    mathematical operations that take two or more operations. These include: +, -, 
    *, /, \^{}, Differentiate, and Integrate. Additionally, each line of a CAT program
    corresponds to a separate expression that will be evaluated in sequence. Thus,
    a newline can be understood as a sequence operator. Assignments are also combining
    forms that bind a variable to some other expression. These are the only kinds of
    statements that interact with future lines (i.e. the line $z = x + y$ would affect
    all future uses of the variable $z$ in subsequent lines).
    \item 
        \begin{enumerate}[label= (\Alph*)]
            \item CAT programs are executed by passing them to the interpreter within
            a text file. For example, if I wanted to run the program \texttt{math.cat},
            I would run \texttt{dotnet run math.cat}.
            \item Evaluating a CAT program will result in two outputs representing the
            same calculation. For each line in a CAT program, the language will attempt
            to simplify the mathematical expression as much as it can. This simplification
            occurs in a sequence of steps. Each step will be printed as a textual output.
            Additionally, a \LaTeX{} file will be produced (and hopefully automatically
            compiled into a pdf) representing the same calculation but in an easier to
            read form.

            For example, the program: \\
            \texttt{(x+1)(x-1) - x\^{}2 \\ Differentiate x\^{}2 + y\^{}2 wrt x} \\
            would produce the output: \\
            \texttt{Simplifying: (x+1)(x-1) - x\^{}2 \\
                ==> x\^{}2 + x - x - 1 - x\^{}2 \\
                ==> 1
            }
            
            \bigskip
            \texttt{Simplifying: Differentiate x\^{}2 + y\^{}2 wrt x \\
                ==> 2x + 0 \\
                ==> 2x
            }
        \end{enumerate}
\end{enumerate}

% DO NOT DELETE ANYTHING BELOW THIS LINE
\end{document}
