\documentclass[10pt]{article}

% Lines beginning with the percent sign are comments
% This file has been commented to help you understand more about LaTeX

% DO NOT EDIT THE LINES BETWEEN THE TWO LONG HORIZONTAL LINES

%---------------------------------------------------------------------------------------------------------

% Packages add extra functionality.
\usepackage{times,graphicx,epstopdf,fancyhdr,amsfonts,amsthm,amsmath,algorithm,algorithmic,xspace,hyperref}
\usepackage[left=1in,top=1in,right=1in,bottom=1in]{geometry}
\usepackage{sect sty}	%For centering section headings
\usepackage{enumerate}	%Allows more labeling options for enumerate environments 
\usepackage{epsfig}
\usepackage[space]{grffile}
\usepackage{booktabs}
\usepackage{forest}
\usepackage{enumitem}   
\usepackage{fancyvrb}
\usepackage{todonotes}

% This will set LaTeX to look for figures in the same directory as the .tex file
\graphicspath{.} % The dot means current directory.

\pagestyle{fancy}

\lhead{Final Project}
\rhead{\today}
\lfoot{CSCI 334: Principles of Programming Languages}
\cfoot{\thepage}
\rfoot{Spring 2024}

% Some commands for changing header and footer format
\renewcommand{\headrulewidth}{0.4pt}
\renewcommand{\headwidth}{\textwidth}
\renewcommand{\footrulewidth}{0.4pt}

% These let you use common environments
\newtheorem{claim}{Claim}
\newtheorem{definition}{Definition}
\newtheorem{theorem}{Theorem}
\newtheorem{lemma}{Lemma}
\newtheorem{observation}{Observation}
\newtheorem{question}{Question}

\setlength{\parindent}{0cm}

%---------------------------------------------------------------------------------------------------------

% DON'T CHANGE ANYTHING ABOVE HERE

% Edit below as instructed

\usepackage{array}

\title{CAT Language Specification} % Replace SnappyLanguageName with your project's name

\author{Michael Faulkner} % Replace these with real partner names.

\begin{document}
  
\maketitle

\subsection*{Introduction}

CAT (Calculus and Algebra Tutor) is a domain-specific language designed to assist 
individuals with mathematical calculations. More specifically, the language allows 
users to input mathematical expressions and operations to perform on the expressions. 
Running a program in this language will then produce a textual represenation of the 
ensuing calculation as well as an optional \LaTeX{} file and PDF that displays the calculation
in a pretty manner.

\bigskip
While there exist computational tools for generalized mathematical expressions 
(for example Mathematica), such tools tend to focus only on getting to an answer. 
The purpose of CAT is to assist its users in understanding \textit{how} to get 
the answer by outputting the intermediate steps of the calculation. This 
addresses another point of annoyance with tools like Mathematica that will 
sometimes output, seemingly arbitrary, restrictions on the parameters alongside 
its final expression. By working through the intermediate steps of the calculation, 
CAT should allow its users to understand why such restrictions were imposed and 
what the limits of the calculation would be if such assumptions were not made.

\subsection*{Design Principles}

The primary goal of CAT is to produce easily understandable calculations. 
This goal corresponds with a prioritization of simpler techniques that are
easier to follow over more powerful techniques that might be able to find
more general answers but are less decipherable. This likely also means a
tradeoff in terms of speed as it is necessary to work through the intermediate
details of a calculation sufficiently to make them human readable.

\subsection*{Examples}

\todo[inline]{Delete this TODO and replace with 3+ examples and accompanying descriptions.}

\subsection*{Language Concepts}

At it's core, CAT is a way of expressing and manipulating mathematical expressions. 
Each line of a CAT program constitutes an expression that will be evaluated. Evaluating
an expression involves reducing it to the simplist form possible by applying operators
and performing algebraic manipulations. The primitives of CAT are variables (i.e. $x, y, z$) 
and numbers (i.e. $1, -10, 3.141592$). These primitives can be combined with mathematical
operations such as +, -, *, /, \^{}, $\sin$, $\cos$, $\log$, sqrt, Differentiate,
and Integrate in order to form expressions. Additionally, users are able to define variables
(i.e. $a = \text{sqrt}(2)$, or $f=x+1$) that can be used in later lines.

\subsection*{Formal Syntax}

\begin{tabular}{>{\texttt\bgroup}l<{\egroup} >{\texttt\bgroup}r<{\egroup} >{\texttt\bgroup}l<{\egroup}}
    <expr> &::=& <number> \\
    <number> &::=& <digits> \\
    &|& <digits>.<digits> \\
    <digits> &::=& <d><digits> \\
    &|& <d> \\
    <d> &::=& 0 | 1 | 2 | 3 | 4 | 5 | 6 | 7 | 8 | 9
\end{tabular}

\subsection*{Semantics}

\todo[inline]{Delete this TODO and replace with as much text as is needed.}

% DO NOT DELETE ANYTHING BELOW THIS LINE
\end{document}
