\documentclass[10pt]{article}

% Lines beginning with the percent sign are comments
% This file has been commented to help you understand more about LaTeX

% DO NOT EDIT THE LINES BETWEEN THE TWO LONG HORIZONTAL LINES

%---------------------------------------------------------------------------------------------------------

% Packages add extra functionality.
\usepackage{times,graphicx,epstopdf,fancyhdr,amsfonts,amsthm,amsmath,algorithm,algorithmic,xspace,hyperref}
\usepackage[left=1in,top=1in,right=1in,bottom=1in]{geometry}
\usepackage{sect sty}	%For centering section headings
\usepackage{enumerate}	%Allows more labeling options for enumerate environments 
\usepackage{epsfig}
\usepackage[space]{grffile}
\usepackage{booktabs}
\usepackage{forest}
\usepackage{enumitem}   
\usepackage{fancyvrb}
\usepackage{todonotes}

% This will set LaTeX to look for figures in the same directory as the .tex file
\graphicspath{.} % The dot means current directory.

\pagestyle{fancy}

\lhead{Final Project}
\rhead{\today}
\lfoot{CSCI 334: Principles of Programming Languages}
\cfoot{\thepage}
\rfoot{Spring 2024}

% Some commands for changing header and footer format
\renewcommand{\headrulewidth}{0.4pt}
\renewcommand{\headwidth}{\textwidth}
\renewcommand{\footrulewidth}{0.4pt}

% These let you use common environments
\newtheorem{claim}{Claim}
\newtheorem{definition}{Definition}
\newtheorem{theorem}{Theorem}
\newtheorem{lemma}{Lemma}
\newtheorem{observation}{Observation}
\newtheorem{question}{Question}

\setlength{\parindent}{0cm}

%---------------------------------------------------------------------------------------------------------

% DON'T CHANGE ANYTHING ABOVE HERE

% Edit below as instructed

\usepackage{array, longtable}

\title{CAT Language Specification} % Replace SnappyLanguageName with your project's name

\author{Michael Faulkner}

\begin{document}
  
\maketitle

\subsection*{Introduction}

CAT (Calculatron and Algebra Tutor) is a domain-specific language designed to assist 
individuals with mathematical calculations. More specifically, the language allows 
users to input mathematical expressions and operations to perform on the expressions. 
Running a program in this language will then produce a textual represenation of the 
ensuing calculation as well as a \LaTeX{} file that can be compiled into a PDF that 
displays the calculation in a pretty manner.

\bigskip
While there exist computational tools for generalized mathematical expressions 
(for example Mathematica), such tools tend to focus only on getting to an answer. 
The purpose of CAT is to assist its users in understanding \textit{how} to get 
the answer by outputting the intermediate steps of the calculation. This 
addresses another point of annoyance with tools like Mathematica that will 
sometimes output, seemingly arbitrary, restrictions on the parameters alongside 
its final expression. By working through the intermediate steps of the calculation, 
CAT should allow its users to understand why such restrictions were imposed and 
what the limits of the calculation would be if such assumptions were not made.

\subsection*{Video Presentation}
View the video \underline{\href{https://youtu.be/nhKoWXYoXG4}{here}} for a video explanation
of CAT! If the above link doesn't work you can type the following into your browser:
\href{https://youtu.be/nhKoWXYoXG4}{https://youtu.be/nhKoWXYoXG4}

\subsection*{Design Principles}

The primary goal of CAT is to produce easily understandable calculations. 
This goal corresponds with a prioritization of simpler techniques that are
easier to follow over more powerful techniques that might be able to find
more general answers but are less decipherable. This likely also means a
tradeoff in terms of speed as it is necessary to work through the intermediate
details of a calculation sufficiently to make them human readable.

\subsection*{Examples}
The following examples are included in a folder called \texttt{examples}. You can run
these examples using the interpreter. For example, to run example 1, you can
use the command \texttt{dotnet run examples/example1.cat} at the project level.
Two things will happen when you run a program: the simplification process will
be printed out as textual output, and a \LaTeX{} file will be created in the same
directory with the corresponding name (i.e. running the above command will
produce the file \texttt{example1.tex} in the examples directory). You can compile
the \LaTeX{} file into a pdf using whatever your standard \LaTeX{} compiler is
(for example, pdflatex).

\begin{enumerate}
    \item The following example illustrates how each line of a CAT program is
    a separate expression to be simplified. The first line here simplifies to
    simplify to $2$, and the second line simplifies to $-1$ .

    \bigskip
    Program: \\
    \texttt{((1 + 3\^{}2) / 2 - 1)\^{}(1/2) \\ (x+1)(x-1) - x\^{}2}

    \bigskip
    Textual Output: \\
    \texttt{Expanding: ((1 + 3\^{}2)2\^{}(-1) + (-1)1)\^{}((1)2\^{}(-1)) \\
        ==> ((-1)1 + (1 + 3\^{}2)2\^{}(-1))\^{}((1)2\^{}(-1)) \\
        ==> ((-1)1 + (1 + (3)3)2\^{}(-1))\^{}((1)2\^{}(-1)) \\
        ==> ((-1)1 + (1)2\^{}(-1) + 3(3)2\^{}(-1))\^{}((1)2\^{}(-1)) \\
        Simplifying: ((-1)1 + (1)2\^{}(-1) + 3(3)2\^{}(-1))\^{}((1)2\^{}(-1)) \\
        ==> ((-1)1 + (1)2\^{}(-1) + 3(3)2\^{}(-1))\^{}((1)2\^{}(-1)) \\
        ==> (-1 + (1)2\^{}(-1) + 3(3)2\^{}(-1))\^{}((1)2\^{}(-1)) \\
        ==> (-1 + 3(3)2\^{}(-1) + 2\^{}(-1))\^{}((1)2\^{}(-1)) \\
        ==> (-1 + 0.5 + 3(3)2\^{}(-1))\^{}((1)2\^{}(-1)) \\
        ==> (-1 + 0.5 + 0.5(3)3)\^{}((1)2\^{}(-1)) \\
        ==> (-1 + 0.5 + 4.5)\^{}((1)2\^{}(-1)) \\
        ==> 4\^{}((1)2\^{}(-1)) \\
        ==> 4\^{}2\^{}(-1) \\
        ==> 4\^{}0.5 \\
        ==> 2
    }

    \bigskip
    \texttt{Expanding: (x + 1)(x + (-1)1) + (-1)x\^{}2 \\
        ==> (1 + x)(x + (-1)1) + (-1)x\^{}2 \\
        ==> (-1)(1)1 + (-1)1x + 1x + xx + (-1)x\^{}2 \\
        ==> (-1)(1)1 + (-1)1x + (-1)xx + 1x + xx \\
        Simplifying: (-1)(1)1 + (-1)1x + (-1)xx + 1x + xx \\
        ==> (-1)(1)1 + (-1)1x + (-1)xx + 1x + xx \\
        ==> -1 + (-1)1x + (-1)xx + 1x + xx \\
        ==> -1 + (-1)x + (-1)xx + 1x + xx \\
        ==> -1 + (-1)x + 1x + xx + (-1)x\^{}2 \\
        ==> -1 + x + (-1)x + xx + (-1)x\^{}2 \\
        ==> -1 + x + (-1)x + (-1)x\^{}2 + x\^{}2 \\
        ==> -1 + 0 + 0 \\
        ==> -1 + 0 \\
        ==> -1
    } 

    \bigskip
    \LaTeX{} Output:
    \subsection*{Expression: $((1 + 3^{2})2^{-1} + (-1)1)^{(1)2^{-1}}$}
    Expanding: $((1 + 3^{2})2^{-1} + (-1)1)^{(1)2^{-1}}$
    $$((-1)1 + (1 + 3^{2})2^{-1})^{(1)2^{-1}}$$
    $$((-1)1 + (1 + (3)3)2^{-1})^{(1)2^{-1}}$$
    $$((-1)1 + (1)2^{-1} + 3(3)2^{-1})^{(1)2^{-1}}$$
    Simplifying: $((-1)1 + (1)2^{-1} + 3(3)2^{-1})^{(1)2^{-1}}$
    $$((-1)1 + (1)2^{-1} + 3(3)2^{-1})^{(1)2^{-1}}$$
    $$(-1 + (1)2^{-1} + 3(3)2^{-1})^{(1)2^{-1}}$$
    $$(-1 + 3(3)2^{-1} + 2^{-1})^{(1)2^{-1}}$$
    $$(-1 + 0.5 + 3(3)2^{-1})^{(1)2^{-1}}$$
    $$(-1 + 0.5 + 0.5(3)3)^{(1)2^{-1}}$$
    $$(-1 + 0.5 + 4.5)^{(1)2^{-1}}$$
    $$4^{(1)2^{-1}}$$
    $$4^{2^{-1}}$$
    $$4^{0.5}$$
    $$2$$
    \subsection*{Expression: $(x + 1)(x + (-1)1) + (-1)x^{2}$}
    Expanding: $(x + 1)(x + (-1)1) + (-1)x^{2}$
    $$(1 + x)(x + (-1)1) + (-1)x^{2}$$
    $$(-1)(1)1 + (-1)1x + 1x + xx + (-1)x^{2}$$
    $$(-1)(1)1 + (-1)1x + (-1)xx + 1x + xx$$
    Simplifying: $(-1)(1)1 + (-1)1x + (-1)xx + 1x + xx$
    $$(-1)(1)1 + (-1)1x + (-1)xx + 1x + xx$$
    $$-1 + (-1)1x + (-1)xx + 1x + xx$$
    $$-1 + (-1)x + (-1)xx + 1x + xx$$
    $$-1 + (-1)x + 1x + xx + (-1)x^{2}$$
    $$-1 + x + (-1)x + xx + (-1)x^{2}$$
    $$-1 + x + (-1)x + (-1)x^{2} + x^{2}$$
    $$-1 + 0 + 0$$
    $$-1 + 0$$
    $$-1$$

    \item This second example highlights CAT's ability to deal with simplifying deeply
    nested expressions and ultimately undwinding all of the complexity to get a simple
    answer. Here, we have an expression of the form $1-x^z$ where $z$ is a complicated
    expression. But $z$ ultimately evaluates to 0, so the overall expression will 
    evaluate to 0.

    \bigskip
    Program: \\
    \texttt{1 - x\^{}(1-x\^{}2/((x + 1)(x - 1) + 1))}

    \bigskip
    Textual Output: \\
    \texttt{Expanding: 1 + (-1)x\^{}(1 + (-1)((x\^{}2)((x + 1)(x + (-1)1) + 1)\^{}(-1))) \\
    ==> 1 + (-1)x\^{}(1 + (-1)(x\^{}2)(1 + (1 + x)(x + (-1)1))\^{}(-1)) \\
    ==> 1 + (-1)x\^{}(1 + (-1)xx(1 + (1 + x)(x + (-1)1))\^{}(-1)) \\
    ==> 1 + (-1)x\^{}(1 + (-1)xx(1 + (-1)(1)1 + (-1)1x + 1x + xx)\^{}(-1)) \\
    ==> 1 + (-1)(x\^{}1)x\^{}((-1)xx(1 + (-1)(1)1 + (-1)1x + 1x + xx)\^{}(-1)) \\
    Simplifying: 1 + (-1)(x\^{}1)x\^{}((-1)xx(1 + (-1)(1)1 + (-1)1x + 1x + xx)\^{}(-1)) \\
    ==> 1 + (-1)(x\^{}1)x\^{}((-1)xx(1 + (-1)(1)1 + (-1)1x + 1x + xx)\^{}(-1)) \\
    ==> 1 + (-1)(x\^{}1)x\^{}((-1)xx(-1 + 1 + (-1)1x + 1x + xx)\^{}(-1)) \\
    ==> 1 + (-1)(x\^{}1)x\^{}((-1)xx(-1 + 1 + (-1)x + 1x + xx)\^{}(-1)) \\
    ==> 1 + (-1)(x\^{}1)x\^{}((-1)xx(-1 + 1 + x + (-1)x + xx)\^{}(-1)) \\
    ==> 1 + (-1)(x\^{}1)x\^{}((-1)xx(-1 + 1 + x + (-1)x + x\^{}2)\^{}(-1)) \\
    ==> 1 + (-1)(x\^{}1)x\^{}((-1)xx(-1 + 0 + 1 + x\^{}2)\^{}(-1)) \\
    ==> 1 + (-1)(x\^{}1)x\^{}((-1)xxx\^{}2\^{}(-1)) \\
    ==> 1 + (-1)(x\^{}1)x\^{}((-1)x\^{}2\^{}(-1 + 1)) \\
    ==> 1 + (-1)(x\^{}1)x\^{}((-1)x\^{}2\^{}0) \\
    ==> 1 + (-1)(x\^{}1)x\^{}((-1)1) \\
    ==> 1 + (-1)(x\^{}(-1))x\^{}1 \\
    ==> 1 + (-1)x\^{}(-1 + 1) \\
    ==> 1 + (-1)x\^{}0 \\
    ==> 1 + (-1)1 \\
    ==> -1 + 1 \\
    ==> 0}

    \bigskip
    \LaTeX{} Output:
    \subsection*{Expression: $1 + (-1)x^{1 + (-1)(x^{2}((x + 1)(x + (-1)1) + 1)^{-1})}$}
    Expanding: $1 + (-1)x^{1 + (-1)(x^{2}((x + 1)(x + (-1)1) + 1)^{-1})}$
    $$1 + (-1)x^{1 + (-1)x^{2}(1 + (1 + x)(x + (-1)1))^{-1}}$$
    $$1 + (-1)x^{1 + (-1)xx(1 + (1 + x)(x + (-1)1))^{-1}}$$
    $$1 + (-1)x^{1 + (-1)xx(1 + (-1)(1)1 + (-1)1x + 1x + xx)^{-1}}$$
    $$1 + (-1)x^{1}x^{(-1)xx(1 + (-1)(1)1 + (-1)1x + 1x + xx)^{-1}}$$
    Simplifying: $1 + (-1)x^{1}x^{(-1)xx(1 + (-1)(1)1 + (-1)1x + 1x + xx)^{-1}}$
    $$1 + (-1)x^{1}x^{(-1)xx(1 + (-1)(1)1 + (-1)1x + 1x + xx)^{-1}}$$
    $$1 + (-1)x^{1}x^{(-1)xx(-1 + 1 + (-1)1x + 1x + xx)^{-1}}$$
    $$1 + (-1)x^{1}x^{(-1)xx(-1 + 1 + (-1)x + 1x + xx)^{-1}}$$
    $$1 + (-1)x^{1}x^{(-1)xx(-1 + 1 + x + (-1)x + xx)^{-1}}$$
    $$1 + (-1)x^{1}x^{(-1)xx(-1 + 1 + x + (-1)x + x^{2})^{-1}}$$
    $$1 + (-1)x^{1}x^{(-1)xx(-1 + 0 + 1 + x^{2})^{-1}}$$
    $$1 + (-1)x^{1}x^{(-1)xx(x^{2})^{-1}}$$
    $$1 + (-1)x^{1}x^{(-1)(x^{2})^{-1 + 1}}$$
    $$1 + (-1)x^{1}x^{(-1)(x^{2})^{0}}$$
    $$1 + (-1)x^{1}x^{(-1)1}$$
    $$1 + (-1)x^{-1}x^{1}$$
    $$1 + (-1)x^{-1 + 1}$$
    $$1 + (-1)x^{0}$$
    $$1 + (-1)1$$
    $$-1 + 1$$
    $$0$$

    \item This last example highlights CAT's current ability to factor expressions.
    CAT automatically expands expressions before simplifying them, so this example
    demonstrates how sometimes CAT will not recover the original expression. In a
    perfect world, this expression would simplify to \texttt{(x + y + z)\^{}2}. But
    after first choosing to factor out the two, CAT gets stuck trying to simplify this
    expression any further.

    \bigskip
    Program: \\
    \texttt{abcd + bdef \\ (x+y+z)\^{}2}

    \bigskip
    Textual Output: \\
    \texttt{Expanding: abcd + bdef \\
        ==> abcd + bdef \\
        Simplifying: abcd + bdef \\
        ==> abcd + bdef \\
        ==> bd(ac + ef)
    }

    \bigskip
    \texttt{
        Expanding: (x + y + z)\^{}2 \\
        ==> (x + y + z)\^{}2 \\
        ==> (z + y + x)(z + y + x) \\
        ==> xx + xy + xy + yy + xz + xz + yz + yz + zz \\
        Simplifying: xx + xy + xy + yy + xz + xz + yz + yz + zz \\
        ==> xx + xy + xy + yy + xz + xz + yz + yz + zz \\
        ==> xy + xy + yy + xz + xz + yz + yz + zz + x\^{}2 \\
        ==> xy + xy + xz + xz + yz + yz + zz + x\^{}2 + y\^{}2 \\
        ==> xy + xy + xz + xz + yz + yz + x\^{}2 + y\^{}2 + z\^{}2 \\
        ==> 2xy + 2xz + 2yz + x\^{}2 + y\^{}2 + z\^{}2 \\
        ==> 2(xy + xz + yz) + x\^{}2 + y\^{}2 + z\^{}2 \\
        ==> 2(yz + x(y + z)) + x\^{}2 + y\^{}2 + z\^{}2
    }

    \bigskip
    \LaTeX{} Output:
    \subsection*{Expression: $abcd + bdef$}
    Expanding: $abcd + bdef$
    $$abcd + bdef$$
    Simplifying: $abcd + bdef$
    $$abcd + bdef$$
    $$bd(ac + ef)$$
    \subsection*{Expression: $(x + y + z)^{2}$}
    Expanding: $(x + y + z)^{2}$
    $$(x + y + z)^{2}$$
    $$(z + y + x)(z + y + x)$$
    $$xx + xy + xy + yy + xz + xz + yz + yz + zz$$
    Simplifying: $xx + xy + xy + yy + xz + xz + yz + yz + zz$
    $$xx + xy + xy + yy + xz + xz + yz + yz + zz$$
    $$xy + xy + yy + xz + xz + yz + yz + zz + x^{2}$$
    $$xy + xy + xz + xz + yz + yz + zz + x^{2} + y^{2}$$
    $$xy + xy + xz + xz + yz + yz + x^{2} + y^{2} + z^{2}$$
    $$2xy + 2xz + 2yz + x^{2} + y^{2} + z^{2}$$
    $$2(xy + xz + yz) + x^{2} + y^{2} + z^{2}$$
    $$2(yz + x(y + z)) + x^{2} + y^{2} + z^{2}$$
\end{enumerate}

\subsection*{Language Concepts}

At it's core, CAT is a way of expressing and manipulating mathematical expressions. 
Each line of a CAT program constitutes an expression that will be evaluated. Evaluating
an expression involves reducing it to the simplist form possible by applying operators
and performing algebraic manipulations. The primitives of CAT are variables (i.e. $x, y, z$) 
and numbers (i.e. $1, -10, 3.141592$). These primitives can be combined with mathematical
operations such as +, -, *, /, and \^{} in order to form expressions. Each CAT program
must end with a \texttt{.cat} extension, and there must not be any trailing newlines or
the parser will be upset that you are asking it to simplify an empty expression.

\bigskip
When you run a program in the interpreter, the program will print out the series of
simplifications that it performs on each expression you gave it. CAT begins simplifying
an expression by first expanding it, so that terms are fully distributed. Then, CAT
begins combining terms and will attempt to refactor the remaining expression into the
simplist form it can. In addition to printing these steps out to \texttt{stdout}, the
interpreter will also create a \LaTeX{} file with the same name as the input program
containing another represenation of the calculation that is slightly easier to read.
The \LaTeX{} file should be easily compiliable using your prefered \LaTeX{} compiler.

\subsection*{Formal Syntax}

\begin{tabular}{>{\texttt\bgroup}l<{\egroup} >{\texttt\bgroup}r<{\egroup} >{\texttt\bgroup}l<{\egroup}}
    <instruction> &::=& <expr> (`\textbackslash{}n' <expr>)\textsuperscript{*}\\
    <expr> &::=& <ws>\textsuperscript{*}<operation><ws>\textsuperscript{*} \\
    &|& <ws>\textsuperscript{*}<parens><ws>\textsuperscript{*} \\
    &|& <ws>\textsuperscript{*}<literal><ws>\textsuperscript{*} \\
    <operation> &::=& <expr><op><expr> \\
    &|& <expr><ws>\textsuperscript{+}<expr> \\
    &|& <expr><parens> \\
    &|& <parens><expr> \\
    &|& -<expr> \\
    <op> &::=& + | - | * | / | \^{} \\
    <parens> &::=& (<expr>) \\
    <literal> &::=& <number> \\
    &|& <variable> \\
    <number> &::=& <digits> \\
    &|& -<digits> \\
    &|& <digits>.<digits> \\
    &|& -<digits>.<digits> \\
    &|& .<digits> \\
    &|& -.<digits> \\
    &|& <digits>. \\
    &|& -<digits>. \\
    <digits> &::=& <d><digits> \\
    &|& <d> \\
    <d> &::=& 0 | 1 | 2 | 3 | 4 | 5 | 6 | 7 | 8 | 9 \\
    <variable> &::=& a | b | c |\dots| x | y | z \\
    <ws> &::=& Any non-newline whitespace character.
\end{tabular}

\subsection*{Semantics}

\begin{longtable}{>{\centering}p{.27 \textwidth}|>{\centering}p{.15 \textwidth}|>{\centering}p{.12 \textwidth}|p{.35 \textwidth}}
    \textbf{Syntax} & \textbf{Abstract Syntax} & \textbf{Prec/Assoc} & \centering \textbf{Meaning} \tabularnewline
    \hline
    \texttt{<number>} & Number of double & N/A & \texttt{number} is a primitive. 
    I represent numbers using the F\# double datatype. \\
    \hline
    \texttt{<variable>} & Variable of char & N/A & \texttt{variable} is a primitive. 
    I represent variables using the F\# char datatype. These represent symbolic
    variables that are used in mathematical expressions. \\
    \hline
    \texttt{<expr> (`\textbackslash{}n' <expr>)\textsuperscript{*}} & Sequence of Expression list & 0 / left &
    \texttt{Sequence} consists of a list of expressions to be simplified. Every CAT program has one \texttt{Sequence}
    which consists of each line of the program. After evaluating, each expression within a \texttt{Sequence} is converted
    into a list of expressions that represent progressive simplifications of the original expression. \\
    \hline
    \texttt{<expr> (+ <expr>)\textsuperscript{+}} & Addition of Expression list & 1 / left &
    \texttt{Addition} consists of a list of expressions to be added. Because addition is 
    associative, the addends are stored in a list since their particular order doesn't matter. \\
    \hline
    \texttt{<expr> (- <expr>)\textsuperscript{+}} & Addition of Expression list & 1 / left &
    \texttt{Addition} also represents subtraction. Each term that is subtracted is represented
    with the addition of the term multiplied by negative one. \\
    \hline
    \texttt{<expr> (* <expr>)\textsuperscript{+}} & Multiplication of Expression list & 2 / left &
    \texttt{Multiplication} consists of a list of expressions to be added. Because multiplication is 
    associative, the factors are stored in a list since their particular order doesn't matter. \\
    \hline
    \texttt{<expr>(<ws>\textsuperscript{+}<expr>)\textsuperscript{*}} & Multiplication of Expression list & 2 / left &
    An alternative syntax for \texttt{Multiplication}. When two terms are next to each other they are implicitly multiplied. \\
    \hline
    \texttt{<expr><parens>} & Multiplication of Expression list & 2 / left &
    An alternative syntax for \texttt{Multiplication}. When two terms are next to each other they are implicitly multiplied. \\
    \hline
    \texttt{<parens><expr>} & Multiplication of Expression list & 2 / left &
    An alternative syntax for \texttt{Multiplication}. When two terms are next to each other they are implicitly multiplied. \\
    \hline
    \texttt{-<expr>} & Multiplication of Expression list & 2 / N/A &
    An alternative syntax for \texttt{Multiplication}. In this case we have the implicit multiplication of the expression by
    negative one. \\
    \hline
    \texttt{<expr> (/ <expr>)\textsuperscript{+}} & Multiplication of Expression list & 2 / left &
    \texttt{Multiplication} also represents division. Each term that is divided is represented
    with the multiplication of the term by the multiplicative inverse of the other term (The 
    multiplicative inverse is the number raised to the negative 1 power). \\
    \hline
    \texttt{<expr>\^{}<expr>} & Exponentiation of Expression * Expression & 3 / right &
    \texttt{Exponentiation} consists of two expressions. One represents the base, and the other represents
    the exponent. Note that Exponentiation is not stored as a list because it is not associative.
\end{longtable}

\subsection*{Remaining Work}
Symbolic math systems can have nearly infinite scope. CAT has traded off a focus on computational
power for its clarity in communicating its simplification steps. While there are essentially 
limitless additions that can be made to a computer math system, factoring expressions is clearly
a current weakpoint of CAT. General purpose factoring of expressions is a very hard problem to
solve given all the possible edge cases. Another goal that I had for CAT was to implement Calculus
operations; however, one barrier I hit with trying to implement derivatives is that taking the 
derivative of arbitrary exponential functions requires being able to express logarithms, which
brings us to the last notable area that CAT could be improved. CAT does not support special 
functions such as $\log$, $\sin$, $\sinh$, $\tan^{-1}$, $\exp$, etc. These functions are highly
useful (especially in the context of calculus), so they logically be some of the next things
that should be added to the language as well.

% DO NOT DELETE ANYTHING BELOW THIS LINE
\end{document}
